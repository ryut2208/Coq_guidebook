\documentclass{jsbook}
%\setlength{\textwidth}{\fullwidth}
%\setlength{\evensidemargin}{\oddsidemargin}

\begin{document}
\chapter{準備\label{準備}}
入手方法やインストール方法など
Windows、Mac両対応にしてあります。
\newpage
\section{Coqの入手}

高橋研で使うのはおそらくWindows機なので、このまま読み進めてください。
もし、Macを使っているなら、Windows編を流し読みしながら\pageref{Mac編}ページのMac編へ
\subsection*{Windows編}
まずは公式サイト(\verb|http://coq.inria.fr/|)からインストーラをダウンロードします。
アドレスに飛んでサイトが表示されたら右上にある「Get Coq」をクリック。
その後表示されたページの中腹あたりにある「Binaries」のWindowsのリンクをクリックしてexeファイルをダウンロード。
ダウンロードが終わったらインストーラを実行して特に何も考えず「はい」なり「次へ」をクリックしてインストールを終わらせます。
Windowsの場合はこれで終了です。
Coqを使う準備が整いました。
\subsection*{Mac編\label{Mac編}}
MacでCoqを使う場合には一工夫必要になります。
筆者が2014年5月頃に最新版のCoqをインストールした時には起動すらすることができませんでした。
そのため、公式サイトからリンクされている最新版dmgパッケージではないパッケージを入手する必要があります。
入手する方法は2つありますが、どちらの方法でやるべきかは断言できません。
これらのどちらかで起動するCoqがインストールできるはずです。
もし、できなかった場合は諦めてWindowsで作業しましょう。
PCが支給されるはずなので問題はないはずです。

\subsubsection*{MacPortsを用いた最新版の入手}
MacPortsを利用して開発版を入手することができます。
開発版のため動作が不安定な場合があるかもしれませんが、重大なバグ等が取り除かれ起動する場合があります。

MacPortsの概要・インストール手順等はさておき\footnote{付録で説明するかもしれないし、自分で調べてもらうかもしれない}、
Coqのインストールの説明をします。
まず、アプリケーション>ユーティリティの中にあるターミナルを起動します。
起動したら下記のコマンドを打ち込みます。
\begin{verbatim}
$ sudu port -v install coq
\end{verbatim}
Enterを押すと本体のパスワードが要求され、ビルドが実行されるはずです。

\subsubsection*{旧バージョンの入手}
公式サイトから安定版の過去バージョンがリンクされています。
最新版のダウンロードページの下にある「Previous and Development versions of Coq」の中でリンクされている「version 8.3」をクリックしてください。
Coq 8.3のダウンロードページが表示されたら、「coq-8.3pl5.dmg」と「coqide-8.3pl5.dmg」をダウンロードします。
ダウンロードできたらパッケージの説明にしたがってインストールします。

\newpage
\section{前提とする知識など}
\subsection*{知識}
\begin{itemize}
\item 論理学の知識
\item プログラミング実習1~3までの知識
\end{itemize}

\subsection*{参考にしたサイト}
これらで紹介されている知識なども本文中に使用している場合があります。
すべてを見終える必要はないので、平行に、参考にする程度に見てください。
場合によっては紹介したサイトの方がわかりやすい部分があるかもしれません。
使い分けてください。
\begin{itemize}
\item プログラミングCoq\\
\verb|http://www.iij-ii.co.jp/lab/techdoc/coqt/|\\
おそらく一番参考にしたサイトです。ここと呼称を同じにしたものがいくつかあります。
\item ソフトウェアの基礎\\
\verb|http://proofcafe.org/sf/|
\end{itemize}
\end{document}